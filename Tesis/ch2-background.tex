\chapter{Marco teórico}\label{ch:background}

Un grafo (o red) simple es una dupla $G = (V, E)$ que consiste de un conjunto de vértices $V$ y un conjunto de aristas $E = \{ \{u, v\} : u, v \in E, u \neq v\}$. Un sistema basado en agentes consiste de un conjunto de entidades heterogéneas, llamadas agentes, que interactúan entre sí acorde a un sistema de reglas. A través de estas interacciones emergen fenómenos a nivel del sistema. Una red o grafo multi-agente es una red donde consideramos a cada nodo como un agente dentro de un sistema basado en agentes. 
Se trabajará con modelos epidemiológicos de compartimentos, a tiempo continuo. En estos se tiene una cantidad finita de compartimentos $C_1, \dots, C_q$, y en el tiempo $t$, cada nodo $u \in C_i$ para un solo $i \in 1, \dots, q$. Entre los modelos compartimentales más comunes se encuentran el SI (susceptible $\rightarrow$ infeccioso), SIS (susceptible $\rightarrow$ infeccioso $\rightarrow$ susceptible), SIR (susceptible $\rightarrow$ infeccioso $\rightarrow$ recuperado), SEIR (susceptible $\rightarrow$ expuesto $\rightarrow$ infeccioso $\rightarrow$ recuperado), SIRS (susceptible $\rightarrow$ infectado $\rightarrow$ recuperado $\rightarrow$ susceptible).