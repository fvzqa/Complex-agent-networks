\chapter{Revisión bibliográfica}\label{ch:litreview}

Los modelos matemáticos para el estudio de epidemias han sido estudiados por décadas \citep{bailey_mathematical_1975, britton_stochastic_2010}. Varios buscan predecir el tamaño final de una epidemia con alguna probabilidad, así como otros buscando controlar el contagio \citep{Nowzari_etal_2016}, mientras otros han estudiado el impacto de las medidas de contención de la propagación del virus \citep{fransson_sir_2019}. En el caso específico de las simulaciones multi-agente, además de ser usadas para el estudio de epidemias \citep{Hassin_2021, Hoertel_Blachier_Blanco_etal_2020, Perez_Dragicevic_2009} también se han usado para abordar problemas de transporte \citep{Horl_2017} o finanzas \citep{Samitas2018}. Nuestro gobierno no está exento de los retos que presenta enfrentar una crisis sanitaria de naturaleza epidémica, y tomar la decisión equivocada puede tener costos exorbitantes tanto en materia económica como en vidad humanas \citep{Lipsitch_etal_2011, Maringe_etal_2020, Pasquini_etal_2017}. 
