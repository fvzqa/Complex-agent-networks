\chapter{Este es un apéndice}

\section{Citas bibliográficas}

En princicpio tienes total libertad de incluir tu bibliografía con el entorno {\tt thebibliography} nativo de \LaTeX{} o mediante la herramienta \textsc{Bib}\TeX. En caso de que optes por esto último (recomendado), puedes usar alguno de los archivos {\tt mighelbib.bst} o {\tt mighelnat.bst} incluidos en el paquete {\tt Tesis-FIME}, pues sus diseños están basados en el estilo bibliográfico estándar del español, además de que armoniza con el estilo de tesis provisto por {\tt fime.cls}.

El estilo bibliográfico {\tt mighelbib} es numérico, es decir cita con un número entre corchetes, por ejemplo una cita \verb+\cite{Dan82}+ genera una etiqueta del tipo [13], mientras que el estilo {\tt mighelnat} es tipo autor-año y requiere que el paquete {\tt natbib} sea cargado (sin opciones) para su correcto funcionamiento, cita con el apellido del autor y el año, por ejemplo una cita \verb+\citet{Dan82}+ genera una etiqueta del tipo Dantzig (1982), mientras que una cita \verb+\citep{Dan82}+ genera una etiqueta del tipo (Dantzig, 1982).

Como muestra del estilo, unas citas: un libro clásico de epidemiología matemática en redes \citep{kiss_mathematics_2017}, o un survey como el de \citet{britton_stochastic_2010}.

\section{Comillas}

El objetivo de esta sección era provocar otra página para que se vea el encabezado. Pero aprovechamos para decir que la clase {\tt fime.cls} carga el paquete {\tt babel} con la opción {\tt spanish}, por lo que cambiará automáticamente los dobles signos $<<$ y $>>$ por << y >>. Estas comillas angulares son las correctas en el idioma español, y son las que se usan en la clase {\tt fime.cls}, por lo que se sugiere sean las usadas en el texto cada que quieras <<entrecomillar>> algo.

