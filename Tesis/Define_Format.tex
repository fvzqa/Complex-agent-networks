\usepackage{hyperref}
    \hypersetup{breaklinks=true,colorlinks=true,
        linkcolor=black,citecolor=black,urlcolor=black}

\usepackage{booktabs}
\usepackage{multicol}
\usepackage{longtable}
\usepackage{tabu}
\extrarowsep = 1mm
\usepackage{multirow}

\usepackage[sort&compress,numbers]{natbib}
\usepackage{amsthm}
\usepackage{mathtools}

\usepackage{subcaption}
\usepackage{float}
\usepackage{pst-solides3d}

\usepackage{url}
\usepackage{xcolor}
\usepackage{color}
\usepackage{soul}

\usepackage{import}

\usepackage[titletoc,toc,page]{appendix}

%Add code in latex
\usepackage{listings}
\usepackage{verbatim}
\usepackage{caption}

\usepackage{xr}

%%%%%%%%%%%%%%%%%%%
% Ticks format %
%%%%%%%%%%%%%%%%%%%
\usepackage{tikz}
\tikzset{
      every picture/.prefix style={
        execute at begin picture=\shorthandoff{;}
      }
    }

\usepackage{pgf}
\usetikzlibrary{shadows,petri,calc,trees,positioning,arrows,chains,shapes.geometric,
    decorations.pathreplacing,decorations.pathmorphing,shapes,%
    matrix,shapes.symbols}

\tikzset{
>=stealth',
  punktchain/.style={
    rectangle,
    rounded corners,
    % fill=black!10,
    draw=black, very thick,
    text width=10em,
    minimum height=3em,
    text centered,
    on chain},
  line/.style={draw, thick, <-},
  element/.style={
    tape,
    top color=white,
    bottom color=blue!50!black!60!,
    minimum width=8em,
    draw=blue!40!black!90, very thick,
    text width=10em,
    minimum height=3.5em,
    text centered,
    on chain},
  every join/.style={->, thick,shorten >=1pt},
  decoration={brace},
  tuborg/.style={decorate},
  tubnode/.style={midway, right=2pt},
}


%%%%%%%%%%%%%%%%%%%%%
% Python format %
%%%%%%%%%%%%%%%%%%%%%
% Custom colors
\definecolor{deepblue}{rgb}{0,0,0.5}
\definecolor{deepred}{rgb}{0.6,0,0}
\definecolor{deepgreen}{rgb}{0,0.5,0}
\definecolor{mygreen}{RGB}{28,172,0} % color values Red, Green, Blue
\definecolor{mylilas}{RGB}{170,55,241}

\definecolor{keywords}{RGB}{255,0,90}
\definecolor{comments}{RGB}{0,0,113}
\definecolor{red}{RGB}{160,0,0}
\definecolor{green}{RGB}{0,150,0}



% Python style for highlighting
\newcommand\pythonstyle{\lstset{ %
language=python,                % choose the language of the code
%basicstyle=\footnotesize,   % the size of the fonts that are used for the code
basicstyle=\ttfamily\scriptsize,
numbers=left,                   % where to put the line-numbers
numberstyle=\footnotesize,      % the size of the fonts that are used for the line-numbers
stepnumber=1,                   % the step between two line-numbers. If it is 1 each line will be numbered
numbersep=5pt,                  % how far the line-numbers are from the code
backgroundcolor=\color{white},  % choose the background color. You must add \usepackage{color}
otherkeywords={self},             % Add keywords here
keywordstyle=\color{deepblue},%
emph={MyClass,__init__},          % Custom highlighting
emphstyle=\ttb\color{deepred},    % Custom highlighting style
morekeywords=[2]{1}, keywordstyle=[2]{\color{deepred}},
identifierstyle=\color{black},%
stringstyle=\color{deepgreen},
commentstyle=\color{mygreen},%
showspaces=false,               % show spaces adding particular underscores
showstringspaces=false,         % underline spaces within strings
showtabs=false,                 % show tabs within strings adding particular underscores
frame=single,           % adds a frame around the code
tabsize=2,          % sets default tabsize to 2 spaces
captionpos=b,           % sets the caption-position to bottom
breaklines=true,        % sets automatic line breaking
breakatwhitespace=false,    % sets if automatic breaks should only happen at whitespace
escapeinside={\%*}{*)}          % if you want to add a comment within your code
}}

\usepackage{newfloat}
\DeclareFloatingEnvironment[
    fileext=los,
    listname=Lista de C\'odigos,
    name=C\'odigo,
    placement=tbhp,
    within=section,
]{codigo}

\newtheorem{mydef}{Definition}
\newtheorem{mytheorem}{Theorem}
\usepackage[spanish, onelanguage]{algorithm2e}
% Python environment
\lstnewenvironment{python}[1][]{
  \pythonstyle
  \lstset{#1}
}
{}

% Python for external files
\newcommand\pythonexternal[2][]{{
\pythonstyle
\lstinputlisting[#1]{#2}}}

% Python for inline
\newcommand\pythoninline[1]{{\pythonstyle\lstinline!#1!}}

\iftrue % \iftrue for applying the fix suggestion
\makeatletter
\renewcommand*{\add@accent}[2]{%
  {\ifx#2i\let\bbl@tempa\i\else\let\bbl@tempa#2\relax\fi
%   fix suggestion, insertion of \relax:       ^^^^^^
    \setbox\@tempboxa\hbox{\bbl@tempa%
      \global\mathchardef\accent@spacefactor\spacefactor}%
    \accent#1\bbl@tempa}\spacefactor\accent@spacefactor}%
\makeatother
\fi


\usepackage[doublespacing]{setspace}

\setlength{\footnotesep}{0pt}
\usepackage{pifont}
\usepackage{bm}

\newcommand{\doublesignature}[5]{%
  \parbox{\textwidth}{
    \vspace{1cm}
    #1\\
    \vspace{2cm}

    \parbox{7cm}{
      \centering
      \rule{6cm}{1pt}\\
       #2 \\
       #3
    }
    \hfill
    \parbox{7cm}{
      \centering
      \rule{6cm}{1pt}\\
      #4 \\
      #5
    }
  }
}