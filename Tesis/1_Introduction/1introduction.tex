\chapter{Introducción}

La humanidad ha sido asediada por enfermedades infecciosas a lo largo de la historia. Ejemplos en la era moderna incluyen las epidemias del SARS, MERS, influenza AH1N1, ébola, y en la actualidad, el SARS CoV-2, virus que causa la enfermedad conocida como covid-19. Ante
estas eventualidades, gobiernos de distintos niveles deben adoptar medidas prontas y efectivas para evitar una crisis de salud pública. Sin embargo, es difícil saber el impacto que tendrán las acciones tomadas ante un sistema complejo y dinámico, como lo es la propagación de una enfermedad en la población. Ante la inviabilidad logística, y quizá ética, de ensayar distintas medidas directamente a nivel población, surge la necesidad de realizar ensayos computacionales mediante modelos matemáticos de la enfermedad. La naturaleza aleatoria y evolutiva de los procesos de contagio hace de las simulaciones estocásticas una de las maneras más efectivas de estudiar y predecir el fenómeno.
Las técnicas de simulación multi-agente permiten analizar y cuantificar los efectos de distintas medidas ante la propagación de enfermedades, tales como el distanciamiento social, el uso de cubrebocas, o el aislamiento social, además de interacciones con otros factores como la densidad poblacional, nivel socioeconómico y la calidad de aire. La comprensión de estas diferencias conlleva a una toma de decisiones facilitada y basada en evidencia científica.




\section{Hipótesis}
La aplicación correcta y oportuna de medidas de contingencia puede ayudar a controlar (o eliminar por completo) la propagación de una enfermedad infecciosa. 


\section{Objetivo}
Diseñar, implementar y analizar una simulación multi-agente epidemiológica que permita medir los efectos que tienen distintas medidas de contingencia contra el contagio y propagación de una enfermedad infecciosa.
\section{Estructura de la tesis}
En el capítulo 2 se describen los conceptos importantes para este trabajo. En el capítulo 3 se discute los trabajos relevantes de simulaciones multi-agente de epidemias,  epidemias en redes. En el capítulo 4 se discute paso a paso la metodología con la que se trató el problema. En el capítulo 5 se muestran los resultados obtenidos en la investigación. Por último, en el capítulo 6 se presentan las conclusiones obtenidas.