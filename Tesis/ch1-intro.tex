\chapter{Introducción}

La humanidad ha sido asediada por enfermedades infecciosas a lo largo de la historia. Ejemplos en la era moderna incluyen las epidemias del SARS, MERS, influenza AH1N1 ébola y en la actualidad, el SARS CoV-2, virus que causa la enfermedad conocida como \textit{covid-19}. Ante estas eventualidades, gobiernos de distintos niveles deben adoptar medidas prontas y efectivas para evitar una crisis de salud pública. Sin embargo es difícil saber el impacto que tendrán las acciones tomadas ante un sistema complejo y dinámico, como lo es la propagación de una enfermedad en una población. Ante la inviabilidad logística, y quizá ética, de ensayar distintas medidas directamente a nivel población, surge la necesidad de realizar ensayos computacionales mediante modelos matemáticos de la enfermedad. La naturaleza aleatoria y evolutiva de los procesos de contagio hace de las simulaciones estocásticas una de las maneras más efectivas de estudiar y predecir el fenómeno.

Las técnicas de simulación multi-agente permiten analizar y cuantificar los efectos de distintas medidas de contención ante la propagación de enfermedades, tales como el distanciamiento social, el uso de cubrebocas,  o el aislamiento social, además de interacciones con otros factores como la densidad poblacional, nivel socioeconómico y la calidad del aire. La comprensión de estas diferencias conlleva a una toma de decisiones facilitada y basada en evidencia científica. Aunado a esto, representar las conexiones e interacciones entre nuestros agentes por medio de una red, permite el uso de técnicas matemáticas bien estudiadas de teoría de grafos y sistemas de propagación en redes. Combinando estas dos metodologías, confiamos que las redes complejas multi-agentes son idóneas para la simulación estocástica de epidemias bajo medidas de contención.

\section{Hipótesis y objetivo}
La hipótesis es que la simulación de modelos epidemiológicos por medio de redes multi-agente permite observar y cuantificar el impacto que distintas medidas de contención tienen en la propagación de una enfermedad infecciosa. Esto permitiría una toma de decisiones públicas más informada y con mejores resultados.

El objetivo general es diseñar, implementar y analizar una simulación multi-agente epidemiológica en una red que permita medir los efectos que tienen distintas medidas de contención contra el contagio y propagación de una enfermedad infecciosa. Los objetivos específicos para el presente trabajo son:
\begin{description}
\item[Modelación.] Diseñar una simulación multi-agente en red de un modelo epidemiológico, el cuál permita medir la propagación de una enfermedad infecciosa bajo distintas medidas de contingencia. Específicamente, del uso de cubrebocas, distanciamiento social y aislamiento.

\item[Implementación.] Implementar un prototipo computacional del modelo desarrollado para explorar la magnitud del impacto que distintas medidas de contingencia, o la ausencia de estas, tengan en medidas clave de la propagación de la enfermedad, como lo son porcentaje final de infectados, duración de la epidemia, máxima cantidad de infectados simultáneamente.

\item[Visualización.] Crear visualizaciones de la propagación de la enfermedad en una red acorde a los resultados obtenidos con la implementación del modelo mencionado, así como del impacto de las medidas de contingencia. 
\end{description}


\section{Estructura de la tesis}

En el capítulo \ref{ch:background} se describen los conceptos clave necesarios para la comprensión de este trabajo. En el capítulo \ref{ch:litreview} se realiza una revisión de la literatura relacionada al tema. En el capítulo \ref{ch:method} se describe la metodología empleada para la solución del problema. Por último en el capítulo \ref{ch:results} se muestran los resultados obtenidos de las experimentaciones y en el capítulo \ref{ch:conclusions} se describen las conclusiones obtenidas.

